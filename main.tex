\documentclass[a4paper, 10pt]{article}
\usepackage[brazilian]{babel}
\usepackage[utf8]{inputenc}
\usepackage[T1]{fontenc}
\usepackage[pdftex]{hyperref}
\begin{document}

	\title{Relatório: Backlog}
	\author{Leonardo Soares Santos}
	\date{\today}
	
	\maketitle	
	\section{Corpora}
    	\subsection{Em Português}
        	\subsubsection{Floresta Sintática}
                \paragraph{Artigo: “Floresta Sintática”: A treebank for Portuguese.}
                Chamamos de "Floresta Sintáctica" um conjunto de frases (corpus) analisadas 
            	(morfo) sintaticamente. Como, além da indicação das funções sintácticas, a análise também explicita hierarquicamente 
            	informação relativa à estrutura de constituintes, dizemos que uma frase sintaticamente analisada se parece com uma árvore, donde um conjunto de árvores constitui uma floresta sintáctica (em inglês, treebank).
            	\paragraph{} 
            	O projecto Floresta Sintá(c)tica é uma colaboração entre a Linguateca e o projecto VISL. Contém textos em português (do Brasil e de Portugal) anotados (analisados) automaticamente pelo analisador sintáctico PALAVRAS (Bick 2000) e revistos por linguistas.
            	\paragraph{} 
            	Atualmente, a Floresta Sintá(c)tica tem quatro partes, que diferem quanto ao gênero textual, quanto ao modo (escrito vs falado) e quanto ao grau de revisão linguística: 
            \begin{itemize}
				\item Bosque, totalmente revisto por linguistas;
                \item Selva, parcialmente revista;
                \item Floresta Virgem e a Amazônia, não revistos.
			\end{itemize}
             Junto, todo esse material soma cerca de 261 mil frases (6,7 milhões de palavras) sintaticamente analisadas
                \paragraph{Disponível em:}
                \begin{itemize}
				    \item https://www.linguateca.pt/floresta/principal.html
				    \item Biblioteca NLTK
	            \end{itemize}

            \subsubsection{AC/DC}
            	\paragraph{O projeto AC/DC }
            	(Acesso a corpos/Disponibilização de corpos), iniciado em 1999, surgiu da necessidade de juntar os poucos recursos disponíveis num único ponto na rede e dessa forma facilitar a comparação e a reutilização do material, permitindo ao mesmo tempo acesso a uma ferramenta poderosa de interrogação de corpos, o \href{http://cwb.sourceforge.net/}{Open CWB} (versão nova do IMS corpus workbench), para a qual desenvolvemos esta interface.
				\paragraph{}
				Desde 2000, a anotação dos corpos tem sido feita automaticamente pelo \href{http://visl.sdu.dk/visl/pt/}{PALAVRAS} de Eckhard Bick, e convertida para o "formato AC/DC", descrito pormenorizadamente na página de \href{https://www.linguateca.pt/acesso/anotacao.html}{Anotação}.
				\paragraph{Disponível em:}
				\begin{itemize}
				    \item https://www.linguateca.pt/ACDC/

			    \end{itemize}

            \subsubsection{CTEMPúblico}
            	\paragraph{} 
            	CETEMPúblico (Corpus de Extractos de Textos Electrónicos MCT/Público) é um corpus de aproximadamente 180 milhões de palavras em português europeu, criado pelo projecto \href{https://www.linguateca.pt/proc_comp_port.html}{Processamento computacional do português} (projecto que deu origem à Linguateca) após a assinatura de um protocolo entre o Ministério da Ciência e da Tecnologia (MCT) português e o jornal \href{https://www.publico.pt/}{PÚBLICO} em Abril de 2000.
            	\paragraph{Disponível em:}
            	\begin{itemize}
				    \item https://www.linguateca.pt/CETEMPublico/

			    \end{itemize}
            \subsubsection{CETENFolha}
            	\paragraph{O CETENFolha (Corpus de Extractos de Textos Electrónicos NILC/
            	Folha de S. Paulo)}é um corpus de cerca de 24 milhões de palavras em português brasileiro, criado pelo projecto \href{https://www.linguateca.pt/proc_comp_port.html}{Processamento computacional do português} (projecto que deu origem à Linguateca) com base nos textos do jornal \href{https://www.folha.uol.com.br/}{Folha de S. Paulo} que fazem parte do corpus NILC/São Carlos, compilado pelo \href{http://www.nilc.icmsc.sc.usp.br/}{Núcleo Interinstitucional de Lingüística Computacional} (NILC).
                \paragraph{Disponível em:}
            	\begin{itemize}
				    \item https://www.linguateca.pt/

			    \end{itemize}
            \subsubsection{CHAVE}
            	\paragraph{A colecção CHAVE}
            	é o resultado da participação da Linguateca na organização do CLEF a partir de 2004, que além de potenciar a participação da comunidade envolvida no processamento da língua portuguesa nesta avaliação conjunta internacional, pretende fomentar e disseminar recursos públicos. Veja a página do \href{https://www.linguateca.pt/CLEF/}{CLEF} mantida pela Linguateca
            	\paragraph{Disponível em:}
            	\begin{itemize}
				    \item https://www.linguateca.pt/

			    \end{itemize}
            \subsubsection{PAPEL}
            	\paragraph{O PAPEL}
            	é um recurso criado pela Linguateca a partir do Dicionário PRO de Língua Portuguesa da \href{http://www.portoeditora.pt/}{Porto Editora} através de um protocolo de colaboração com o departamento de dicionários desta empresa.
				\paragraph{}
				Contrário de outras ontologias lexicais para o português de que temos conhecimento, o PAPEL é público, grátis e utilizável por todos os actores de processamento da língua que o quiserem usar, e encontra-se aberto para subsequente melhoria pela comunidade.
				\paragraph{} 
				Desenvolvimento do PAPEL oficial terminou com o fim da terceira fase da Linguateca (31 de Dezembro de 2008). A publicação da sua primeira versão teve a data de 17 de Agosto de 2009, e desde aí temos tentado produzir novas versões através da melhoria e validação do seu conteúdo (veja-se o historial para mais informação).
				\paragraph{Disponível em:}
            	\begin{itemize}
				    \item https://www.linguateca.pt/

			    \end{itemize}

            \subsubsection{Lácio-Web}
            	\paragraph{Lácio-Web (LW)}
            	é um projeto de 30 meses de duração financiado pelo CNPq, iniciado em janeiro de 2002, com parceria entre NILC (Núcleo Interinstitucional de Lingüística Computacional), localizado no ICMC-USP, IME (Instituto de Matemática e Estatística) e FFLCH (Faculdade de Filosofia, Letras e Ciências Humanas).
                \paragraph{}
                O objetivo deste projeto é divulgar e disponibilizar livremente na Web: a) vários córpus do português brasileiro escrito contemporâneo, representando bancos de textos adequadamente compilados, catalogados e codificados em um padrão que possibilite fácil intercâmbio, navegação e análise; e b) ferramentas lingüístico-computacionais, tais como contadores de freqüência, concordanciadores e etiquetadores morfossintáticos. O público-alvo do LW é heterogêneo: de um lado lingüistas, cientistas da computação, lexicógrafos, e etc. e, de outro, não especialistas em geral.
                \paragraph{}O LW comporta 6 corpora:
                \begin{itemize}
                	\item Uma referência para o corpus chamado Lacio-Ref;
                    \item Mac-Morpho, uma porção do Lácio-Ref em padrão ouro, composta 1,1 milhão de palavras, no qual foi manualmente validado por tags morfo-sintáticas;
                    \item Uma porção do Lacio-Ref automaticamente anotado com lemas, POS e tags sintáticas no qual foi utilizado o parser Curupira desenvolvido no Nilc; 
                    \item Um corpus composto de textos não revisados (Lacio-Dev);
                    \item Par-C e Comp-C são textos Potuguês-Inglês.
                \end{itemize}
                \paragraph{Disponível em:}
            	\begin{itemize}
				    \item https://sites.google.com/site/linguacorpus/lacioweb
                    \item https://sites.google.com/site/linguacorpus/
			    \end{itemize}
            \subsubsection{Corpus Brasileiro}
                \paragraph{O projeto Corpus Brasileiro,}
                do visa construir e disponibilizar online o Corpus Brasileiro, que será composto por um bilhão de palavras de português brasileiro comteporâneo, de vários tipos de linguagem.
                \paragraph{Informações:}
            	\begin{itemize}
            	    \item Não classificado
				    \item Disponível em: http://corpusbrasileiro.pucsp.br/cb/Inicial.html
			    \end{itemize}
        \subsection{Outros Idiomas}
        	\subsubsection{MASC}
            	\paragraph{O Manually Annotated Sub-Corpus (MASC)}
            	consiste em aproximadamente 500 mil palavras do inglês americano contemporâneo escritas ou faladas, esse sub-corpus foi gerado a partir do OANC(Open American National Corpus)
                \paragraph{}
                Tudo do MASC inclui anotações manualmente validadas para cada sentença. token, lemma and POS; substantivo e verbos; entidades nomeadas(pessoa, localização, organização, data); Sintaxe do projeto Penn Treebank; co-referência; e estrutura discursiva.
                
                \paragraph{Disponível em:}
                \begin{itemize}
                    \item http://www.anc.org/data/masc/
                \end{itemize}
            \subsubsection{ANC-Second Release}
            	\paragraph{a segunda versão do ANC(American National Corpus),}
            	contém por volta de 22 milhões de palavras do inglês americano escrito e falado, anotado por lema, PoS. PoS tags usando Penn tagset são utilizados para todos os dados da segunda versão, e muitos documentos são também Pos-tagged usando o Biber tagset.
            	
            	\paragraph{Disponível em:}
                \begin{itemize}
                    \item http://www.anc.org/data/anc-second-release/
                \end{itemize}
            \subsubsection{OANC}
            	\paragraph{O OANC(Open American National Corpus)}
            	tem por volta de 15 milhões de palavras advindos da segunda versão do ANC, que são irrestritos em termos de uso e redistribuição. Desde 2006, o projeto ANC tem se comprometido em produzir somente dados abertos. Assim sendo, o OANC e MASC são os únicos recursos que o ANC continuará desenvolvendo.
            	
            	\paragraph{Disponível em:}
                \begin{itemize}
                    \item http://www.anc.org/data/oanc/
                \end{itemize}
            \subsubsection{BNC}
            	\paragraph{O BNC(British National Corpus)}
            	é uma coleção de 100 milhões de palavras de trechos da lingua inglesa escritos e falados de uma grande quantidade de fontes, desenvolvido para representar uma grande parte do inglês britânico da última parte do século 20, tanto falado como escrito. A última edição é \textit{BNC XML EDITION}, lançada em 2007.
                \paragraph{}A parte escrita do BNC (90\%)
                inclui, por exemplo, trechos de jornais regionais e nacionais, periódicos e revistas especializadas para todas as idades e interesses, livros acadêmicos e ficção popular, cartas e memorandos publicados e não publicados, dissertações escolares e universitárias, entre outros tipos de texto. A parte falada (10\%) consiste em transcrições ortográficas de conversas informais não roteirizadas (gravada por voluntários selecionados de diferentes idades, regiões e classes sociais de uma forma balanceada demográficamente) e língua falada em diferentes contextos, variando de encontros de negócios formais até reuniões governamentais para programas de rádio e telefonemas.
                \paragraph{Disponível em:}
                \begin{itemize}
                    \item https://corpus.byu.edu/bnc/
                \end{itemize}
           \subsubsection{The Penn Treebank} 
           		\paragraph{O projeto Penn TreeBank,} 
           		produziu aproximadamente 7 milhões de palavras classificadas com PoS taggers, 3 milhões de palavras esqueleticamente analisadas, por volta de 2 milhões de palavras de texto analisados por estruturas de argumento e predicado, e 1,6 milhão de palavras transcritas de texto falado anotado por \textit{speech disfluencies}. 
                \paragraph{}
                Uma atualização do corpus ocorreu em 1999, adicionando texto comutado, dysfluency-annotated e analisado; Texto analisado do corpus Brown.
                
                \paragraph{Disponível em:}
                \begin{itemize}
                    \item https://catalog.ldc.upenn.edu/ldc99t42
                \end{itemize}
    \section{Taggers}
    	\subsection{Em Português}
        	\subsubsection{PALAVRAS}
            	\paragraph{O PALAVRAS é um analisador automático}
            	(tagger-parser) para português que foi desenvolvido por Eckhard Bick no contexto dum projeto de doutoramento (1994-2000) na Universidade de Århus (Dinamarca). O sistema apoia-se num léxico de 50.000 lemas e milhares de regras gramaticais para fornecer uma análise completa, tanto morfológica como sintática, de qualquer texto. O formalismo aplicado integra-se na tradição da Constraint Grammar (CG), introduzido por Fred Karlsson (Universidade de Helsínquia, Finlândia) em 1992. Embora usando um conjunto de etiquetas gramaticais bastante diversificado, o parser alcança um nível de correção de 99\% em termos de morfologia (classe de palavras e flexão), e 97-98\% em termos de sintaxe.
            \subsubsection{Aelius Portuguese POS-Tagger}
                \paragraph{Aelius}
                é um software livre em desenvolvimento, projeto de código aberto com o objetivo de desenvolver um conjunto de módulos baseados em Python, NLTK (Natural Language Toolkit) e interfaces para ferramentas externas livremente disponíveis para análise superficial do Protuguês Brasileiro. Também inclui recursos linguisticos como modelos linguisticos, trecho de textos e padrões ouro. Atualmente, aelius já oferece suporte para corpora taggeado e classificado, oferece uma uma variedade de formatos para a saída, tais como XML na codificação TEI P5.
                \paragraph{Informações:}
                \begin{itemize}
                    \item Técnica: Desconhecida
                    \item Disponível em: http://aelius.sourceforge.net/
                \end{itemize}
        \subsection{Outros Idiomas}
        	\subsubsection{MAT - Multidimensional Analysis Tagger}
            	\paragraph{O MAT (Multidimensional Analysis Tagger}
            	é um programa para Windows que replica Biber's tagger para análise multidimencional funcional de texto em inglês, geralmente aplicado para estudos de variados tipos e gêneros de textos. O programa gera uma versão gramaticalmente anotada do corpus ou texto selecionado como também as estatísticas necessárias para realizar uma análise de tipo ou gênero de texto. O programa plota a entrada de texto ou corpus no Biber's Dimensions e determina o quão próximo é o tipo de texto, como proposto por Biber. Finalmente, o programa oferece uma ferramenta para visualização das Dimensões de um certo texto.
                \paragraph{}
                Essa é um implementação do tagger usado por Biber e outros trabalhos. Esse tagger tenta replicar a analise do Biber o mais próximo possível levando em conta o algoritmo que o autor apresenta no apêndice do livro. A análise básica é feita através do Stanford Tagger. Este tagger inclui uma cópia do Stanford Tagger na qual é rodado automaticamente para produzir um análise gramatical preliminar. MAT então expande a tag do Stanford Tagger usando o identificador linguistico usados no Biber Tagger. 
                
                \paragraph{Disponível em:}
                \begin{itemize}
                    \item https://sites.google.com/site/multidimensionaltagger/versions
                \end{itemize}
            \subsubsection{Stanford Log-linear Part-Of-Speech Tagger}
            	\paragraph{O Stanford Log-Linear Part-Of-Speech Tagger}
            	segue as seguintes idéias:
                \begin{itemize}
                	\item explicitar o uso de tanto de contexto anterior e posterior da tag por meio de uma representação de uma rede de dependência;
                    \item amplo uso de recursos lexicais, incluindo o condicionamento conjunto de várias palavras consecutivas;
                    \item uso efetivo de prioridades em modelos log-lineares ,e
                    \item uma modelagem refinada de recursos de palavras desconhecidas.
                \end{itemize}
            	Usando essas ideias juntas, o resultado que tagger conseguiu uma precisão de 97,24\% no Penn treebank WSJ, uma redução de erro de 4,4\% do melhor resultado já conseguido por tagger de aprendizagem automática sozinho.
            	
        	    \paragraph{Disponível em:}
                \begin{itemize}
                    \item https://nlp.stanford.edu/software/tagger.shtml
                \end{itemize}
	\section{Técnicas}
    	\subsection{Bidirectional Dependency Network}
        	\paragraph{Esta técnica é utilizada no O Stanford Log-Linear Part-Of-Speech Tagger,}
        	fazendo um resumo, é desenhado uma rede dependencia no qual cada nó tem como vizinhos todos os outros que tenham influência sobre ele diretamente. A vizinhança de cada nó é considerada isoladamente e um modelo local é treinado para maximizar dados treinados a máxima verossimilhança daquele nó. No momento do teste, a sequência com o maior score do produto condicional local é calculado e retornado. Sempre é possível achar a sequência de maximização exata, mas apenas em caso de uma rede aciciclica é garantida ser a sequência de máxima verossimilhança.
	    \subsection{Collobert(2011)}
            \paragraph{}
            Nessa seção será descrito a técnica utilizada no POS-tagging do corpus Mac-Morpho, descrito no artigo Natural Language Processing (Almost) from Scratch (Collobert, 2011).
            \paragraph{}
            A técnica tem como objetivo utilizar uma arquitetura de rede neural única e um algoritmo de aprendizado que possa ser aplicado em vários processos do processamento de linguagem natural incluindo POS-tagging, chunking, reconhecimento de entidades nominais e classificação semântica.
            \paragraph{}
            Um resumo do funcionamento da técnica explicitada no artigo é: os recursos são processado no menor tamanho possível,  E em seguida é usada uma arquitetura de rede neural multicamada (NN), treinada de ponta a ponta. A arquitetura recebe a sentença de entrada e aprende várias camadas de recursos extraídas do processamento da entrada. Os recursos computados pelas camadas profundas da rede são automaticamente treinadas por retropropagação para serem relevantes a tarefa.

    \section{Sites Úteis}
        \begin{itemize}
            \item \href{https://www.sketchengine.eu/}{https://www.sketchengine.eu/}
            \item \href{https://sites.google.com/site/linguacorpus/}{https://sites.google.com/site/linguacorpus/} 
            \item \href{http://corpusbrasileiro.pucsp.br/cb/Inicial.html}{http://corpusbrasileiro.pucsp.br/cb/Inicial.html} 
            \item \href{https://drive.google.com/open?id=1G9of0GN_fJC-7IREvetYhzn4XaQWWojG}{Drive contendo boa parte do conteúdo deste documento}
        \end{itemize}
\end{document}